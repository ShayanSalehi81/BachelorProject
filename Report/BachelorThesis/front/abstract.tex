
% -------------------------------------------------------
%  Abstract
% -------------------------------------------------------


\شروع{وسط‌چین}
\مهم{چکیده}
\پایان{وسط‌چین}
\بدون‌تورفتگی

این پروژه به بررسی قدرت تشخیص اهمیت یک خبر فارسی توسط مدل‌های زبانی بزرگ پرداخته و قدرت یادگیری از محتوا، قدرت استدلال و قدرت تفکر آن را ارزیابی کرده است. در ابتدا، از دادگان علائم‌گذاری‌شده توسط افراد در حوزه‌های مختلف از جمله ورزشی، سیاسی، اجتماعی، پزشکی و فرهنگی استفاده و محیطی برای ارزیابی مدل‌های زبانی بزرگ توسعه داده شده است. در این محیط مدل‌های مختلف موجود بررسی و ارزیابی شده و در نهایت با تمام حالات مختلف و شرایط مخلتف، قدرت تحلیل آنها در زبان فارسی و انگلیسی بررسی شده است. این پروژه نشان‌داده که دستور‌های\پانویس{Prompt} شامل زنجیره تفکر\پانویس{Chain-of-Thoughts} و درخت تفکر\پانویس{Tree-of-Thoughts} باعث بهبود کارایی مدل‌ها و همچنین روش نتظیم نمادها\پانویس{Symbol Tuning} باعث حساسیت بسیار زیاد به پرسش داده شده و محتوای آن می‌شود.

\پرش‌بلند
\بدون‌تورفتگی \مهم{کلیدواژه‌ها}: 
مدل‌های زبانی بزرگ، پردازش زبان‌های طبیعی، یادگیری ماشین، تشخیص اهمیت اخبار
\صفحه‌جدید
