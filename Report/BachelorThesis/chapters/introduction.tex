
\فصل{مقدمه}

در دنیای رو به پیش رفت روزمره، حجم عظیمی از اخبار شبانه‌روز به سمت کاربران روانده می‌شود. در این حین می‌دانیم که بسیاری از این اخبار مبنای درستی نداشته و بسیاری نیز برای کاربران بسیار اهمیت کمی دارد. با معرفی یک بستر که بتوان به وسیله آن اخبار مهم به خصوص با توجه به فرهنگ ایرانیان تشحیص داده خود یک چالش بزرگ اما بسیار کاربردی است. در اینجا با استفاده و بهره‌گیری از مدل‌های زبانی بزرگ و دانش که توسط آنها جمع‌آوری شده است به انجام این امر پرداختیم. در ادامه همچنین چالش‌های این مدل‌ها و منطبق نبودن آن طبق فرهنگ و عادات ایرانیان بررسی می‌کنیم و با ارائه روش یادگیری چند نمونه\پانویس{Few-Shot Learning}، این مشکل را برای طرف می‌کنیم.

\قسمت{تعریف مسئله}

مسئله به این شکل تعریف می‌شود که یک خبر در هر دسته‌ای که قرار داشته باشد یا دارای اهمیت بالا یا برچسب ۱ و یا دارای اهمیت پایین و برچسب ۰ است. با دادگان جمع‌‌آوری شده و برچسب‌گذاری‌های انسانی روی آنها، به ۵۵۰۹ داده آموزش و ۱۱۸۰ داده تست و ازیابی رسیده، که با با استفاده از آنها مدل‌ها توصیه نمونه براساس شباهت تعریف شده است و هدف آن است که مدل بتواند اهمیت خبر (۰ یا ۱) را تشخیص دهد و به کاربر اعلام کند.

\قسمت{اهمیت موضوع}

از اهمیت این کار و محیط توسعه داده‌شده می‌توان به موارد زیر اشاره کرد:

\شروع{فقرات}
\فقره
تسهیل پیگیری اخبار برای کاربران، از آنجایی که این محیط توان تشخیص اخبار مهم در دسته‌ها مختلف را داشته، می‌توان برای کاربران صرفا اخبار مهم را دسته‌بندی کرده و آنها با خواندن این اخبار در وقت خود نسبت به خواندن مطالب بی‌اهمیت صرفه‌جویی خواهند کرد.
\فقره
بررسی قدرت استدلال و تفکر مدل‌های زبانی بزرگ، از آنجایی تشخیص اهمیت یک خبر کار نسبتا پیچیده‌ای برای این مدل‌ها احتساب می‌شود، این بستر فراهم شده است که قدرت استدلال و تحلیل مدل‌های مختلف در شرایط گوناگون ارزیابی و اعلام شود.
\فقره
در این کار، روش‌هایی برای بهبود و بهینه کردن دقت این مدل‌ها پیشنهاد و بررسی شده که در جنبه‌های دیگری غیر از تشخیص اخبار مهم می‌توان کمک کننده باشد و به کار گرفته شود. از جمله اینها مسئله طبقه‌بندی و یادگیری محتوای دستور یا درخواست داده شده به مدل‌های زبانی بزرگ است.
\پایان{فقرات}

\قسمت{ادبیات موضوع}
از آنجایی که بخش‌هایی از این پروژه الهام گرفته و ادامه کار تنظیم نمادها\مرجع{wei2023symboltuningimprovesincontext}
بوده از ادبیات این کار نیز در اینجا استفاده شده است. نتظیم نمادها عبارت است از روشی که به جای برچسب‌های اصلی که در اینجا همان ۰ یا ۱ هستند، یک رشته‌ از نماد‌ها همانند !، \#، \& و کاراکترهای دیگر جایگزین شود و مدل نتواند به دانش پیشینه خود اتکا کند.


\قسمت{اهداف پژوهش}

اهداف این پژوهش صورت گرفته به دو قسمت کلی تقسیم می‌شود:
\شروع{فقرات}
\فقره
ابتدا با ساختار و تعریف اخبار مهم پرداخته شده است، در این پژوهش نتیجه‌ها و بررسی‌های انجام شده حاکی این موضوع است که اخبار در دسته‌های گوناگون و برای اشخاص با فرهنگ‌های مختلف اهمیت متفاوتی دارد. بنابراین انجام یک مسئله طبقه‌بندی روی آنها کار آسانی نبوده و با بهبودهای انجام شده در این پژوهش، مسیری برای پژوهش‌های بعدی در جهت رسیدن که دقت بالا با درنظر گرفتن تمام این شرایط فراهم کند.
\فقره
مدل‌های زبان بزرگ که کانون اصلی توجه این پژوهش بوده است در این مسئله خاص به طور کامل بررسی شده و تمامی نقاط ضعف و قوت این مدل‌ها در تشخیص اهمیت اخبار بررسی شده است. همچنین تفاوت قدرت استدلال این مدل‌ها در زبان فارسی با انگلیسی مورد مقایسه قرار گرفته که خود می‌تواند مورد استناد برای پژوهش‌های آینده در این زمینه قرار گیرد.
\پایان{فقرات}