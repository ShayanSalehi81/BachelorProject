\فصل{کارهای پیشین}

همواره در طول زمان بررسی اهمیت اخبار چه در زبان فارسی و چه در زبان انگلیسی یک دغدغه و یک کار مبهم بوده است. از آنجایی که اهمیت یک خبر وابسته به عوامل مختلف همانند فرهنگ، موقعیت جفرافیایی، سلایق شخصی و دیدگاه‌های کاربران بوده در نگاه اول به نظر این کار، ناممکن می‌رسد. اما پژوهش‌های اخیر نشان‌داده است که با استفاده از دادگان‌های برچسب‌گذاری شده و استفاده از یادگیری چند نمونه‌ای می‌توان به نتایج قابل‌قبولی برای این قسمت رسید.

در اینجا به روش‌های مختلف که در گذشته برای بررسی اهمیت اخبار توسعه‌ داده‌شده است پرداخته شده است و سپس مسیر‌های مختلف بررسی و آنالیز این طبقه‌بندی را در مدل‌های زبانی‌ بزرگ بیان شده است.

\قسمت{تشخیص اهمیت اخبار}
این بخش به بررسی روش‌های مختلفی می‌پردازد که در طول زمان برای تشخیص اهمیت اخبار استفاده شده‌اند. این روش‌ها شامل رویکردهای کلاسیک، یادگیری ماشین و هوش مصنوعی، یادگیری عمیق و در نهایت مدل‌های زبانی بزرگ هستند.

\زیرقسمت{رویکردهای کلاسیک}
در روش‌های کلاسیک، تشخیص اهمیت اخبار بیشتر بر اساس معیارهای دستی انجام می‌شد. از معیارهایی مانند طول خبر، تعداد دفعات ذکر شدن یک موضوع در منابع مختلف، یا تحلیل‌های آماری ساده برای این کار استفاده می‌شد.\مرجع{classicapproach}
این روش‌ها به دلیل محدودیت در قابلیت درک معنایی متون، کارایی پایینی در مسائل پیچیده داشتند.

\زیرقسمت{رویکردهای تشخیص اهمیت با استفاده از یادگیری ماشین}
با ظهور الگوریتم‌های یادگیری ماشین\پانویس{Machine Learning}، از مدل‌هایی مانند ماشین بردار پشتیبان\پانویس{SVM}، دسته‌بند بیزین ساده و جنگل‌های تصادفی\پانویس{Random Forest Tree}\مرجع{article222}
برای تحلیل اخبار و تشخیص اهمیت آن‌ها استفاده شد.\مرجع{mlnewsappr}
این روش‌ها از ویژگی‌های استخراج‌شده مانند تعداد کلمات کلیدی، میزان تعامل خبرها و تعداد نقل‌قول‌ها بهره می‌بردند.

\زیرقسمت{رویکرد یادگیری عمیق}
با توسعه یادگیری عمیق\پانویس{Deep Learning}، استفاده از شبکه‌های عصبی مانند شبکه‌های بازگشتی و شبکه‌های توجه‌محور برای درک معنایی متون و تحلیل اخبار رواج یافت. مدل‌هایی نظیر LSTM و GRU توانستند با درک وابستگی‌های طولانی‌مدت در متن\مرجع{necosssss}، عملکرد چشمگیری ارائه دهند.

\زیرقسمت{استفاده از مدل‌های زبانی بزرگ}
مدل‌های زبانی بزرگ با پیش‌آموزش بر داده‌های گسترده، توانایی تحلیل متون خبری را با دقت بالا فراهم کرده‌اند.\مرجع{devlin2019bertpretrainingdeepbidirectional}
این مدل‌ها با توجه به پیکربندی و حجم داده‌های آموزشی، قادرند وظایف مختلف را به‌صورت چندمنظوره انجام دهند.

در کارهای پیشین انجام شده بررسی شده که رفتار مدل‌های زبانی بزرگ در تشخیص اخبار جعلی چطور بوده است. به خصوص در پژوهش‌های قبلی\مرجع{Hu_2024}
با تعریف بازیگر خوب و بد، سعی بر ارزیابی این نوع اخبار داشته و نشان می‌دهد که این مدل‌ها توانایی مناسب جهت تشخیص اخبار جعلی در شرایط از پیش‌تعریف شده مناسب خواهند داشت.

همچنین کارهای فراتری نسبت به صرفا اتکا کردن به پردازش متن انجام شده است، به طوری که با بهره‌گیری همزمان از مدل‌های تصویری مانند CLIP اهمیت اخبار براساس محتوای تصویری، ویدیوی و صوتی به همراه متن آنها نیز بررسی شود.\مرجع{articleclip}


\قسمت{تنظیم بر اساس دستورالعمل و تنظیم نمادین}
تنظیم بر اساس دستورالعمل به آموزش مدل‌های زبانی بزرگ با داده‌هایی اشاره دارد که شامل دستورالعمل‌های دقیق برای انجام وظایف هستند.\مرجع{sanh2022multitaskpromptedtrainingenables}
این روش باعث می‌شود مدل‌ها بتوانند وظایف مشخصی مانند دسته‌بندی اهمیت اخبار را با دقت بیشتری انجام دهند. از سوی دیگر، تنظیم نمادین\پانویس{Symbol Tuning} شامل استفاده از اطلاعات ساختاریافته مانند نمودارهای دانش یا نمایش‌های معنایی برای تقویت عملکرد مدل‌ها است.


\قسمت{یادگیری چندنمونه‌ای در تشخیص اهمیت اخبار}
در این روش، مدل‌ها با تعداد بسیار کمی از نمونه‌های آموزشی، وظایف خود را یاد می‌گیرند. این ویژگی در تحلیل اخبار و تشخیص اهمیت آن‌ها، به‌ویژه در مواقعی که داده‌های آموزشی محدود است، کاربرد دارد. مدل‌هایی مانند Aya توانسته‌اند نشان دهند که تنها با چند نمونه ورودی می‌توانند وظایف پیچیده را انجام دهند.\مرجع{electronics13040799}


\قسمت{تشخیص اهمیت اخبار فارسی}
در گذشته، تلاش‌هایی برای توسعه مدل‌های تشخیص اهمیت اخبار فارسی صورت گرفته است. این تلاش‌ها بیشتر بر اساس روش‌های یادگیری ماشین کلاسیک بوده و از ویژگی‌های زبانی خاص فارسی مانند ریشه‌یابی و تحلیل صرفی بهره گرفته‌اند.\مرجع{Heydari_2021}
با این حال، استفاده از مدل‌های زبانی بزرگ برای زبان فارسی هنوز در مراحل ابتدایی قرار دارد.


همچنین این پژوهش، ادامه مسیر کار خبرچین\مرجع{hemati2023khabarchinautomaticdetectionimportant}
بوده که با استفاده‌ از مدل‌های مبنی بر معماری ترانسفورمر سعی داشته که به بررسی اهمیت اخبار بپردازد. در این پژوهش سعی شده با استفاده از مدل‌های زبانی بزرگ رویکرد کلی‌تری نسبت به بررسی اهمیت اخبار ارائه شود که بتواند بستر جامع‌‌تری برای طبقه‌بندی این حوزه فراهم کند.


\قسمت{سامانه‌های درخواست وابسته به پرسش}
این سامانه‌ها با طراحی درخواست‌هایی که وابسته به موضوع پرسش هستند، قادرند نتایج بهینه‌ای در تشخیص اهمیت اخبار ارائه دهند. برای این منظور، از مهندسی درخواست استفاده می‌شود تا مدل‌های زبانی بزرگ بتوانند بر اساس متن ورودی و هدف پرسش، خروجی مطلوبی تولید کنند.\مرجع{reynoldspromptprogramm}

پژوهش‌های اخیری در این زمینه انجام شده است که نشان می‌دهد استفاده از دستور‌های مختلف بسته به ورودی کاربر می‌تواند نتایج ثمربخش‌تر به ارقام بیاورد. اگرچه در یکسری پژوهش‌های به این رویکرد یادگیری تقویتی\پانویس{Reinforcement Learning} الحاق می‌شود\مرجع{sunationoptimization}
اما بیشتر یک سیستم در پس‌زمینه بوده که بتواند بهترین دستور را با توجه به ورودی و محتوا تشخیص دهد.