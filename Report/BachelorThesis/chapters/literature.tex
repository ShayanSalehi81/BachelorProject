\فصل{کارهای پیشین}

همواره در طول زمان بررسی اهمیت اخبار چه در زبان فارسی و چه در زبان انگلیسی یک دغدغه و یک کار مبهم بوده است. از آنجایی که اهمیت یک خبر وابسته به عوامل مختلف همانند فرهنگ، موقعیت جفرافیایی، سلایق شخصی و دیدگاه‌های کاربران بوده در نگاه اول به نظر این کار، ناممکن می‌رسد. اما پژوهش‌های اخیر نشان‌داده است که با استفاده از دادگان‌های برچسب‌گذاری شده و استفاده از یادگیری چند نمونه‌ای می‌توان به نتایج قابل‌قبولی برای این قسمت رسید.

در اینجا به روش‌های مختلف که در گذشته برای بررسی اهمیت اخبار توسعه‌ داده‌شده است پرداخته شده است و سپس مسیر‌های مختلف بررسی و آنالیز این طبقه‌بندی را در مدل‌های زبانی‌ بزرگ بیان شده است.

\قسمت{تشخیص اهمیت اخبار}
این بخش به بررسی روش‌های مختلفی می‌پردازد که در طول زمان برای تشخیص اهمیت اخبار استفاده شده‌اند. این روش‌ها شامل رویکردهای کلاسیک، یادگیری ماشین و هوش مصنوعی، یادگیری عمیق و در نهایت مدل‌های زبانی بزرگ هستند.

\زیرقسمت{رویکردهای کلاسیک}
در روش‌های کلاسیک، تشخیص اهمیت اخبار بیشتر بر اساس معیارهای دستی انجام می‌شد. از معیارهایی مانند طول خبر، تعداد دفعات ذکر شدن یک موضوع در منابع مختلف، یا تحلیل‌های آماری ساده برای این کار استفاده می‌شد.\مرجع{classicapproach}
این روش‌ها به دلیل محدودیت در قابلیت درک معنایی متون، کارایی پایینی در مسائل پیچیده داشتند.