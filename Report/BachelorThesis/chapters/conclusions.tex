\فصل{نتیجه‌گیری}
در این پژوهش به بررسی ابعاد مختلف استفاده از مدل‌های زبانی بزرگ در تشخیص اهمیت اخبار پرداخت شد. از بررسی جابه‌جایی برچسب‌ها با مقدار نمادین آنها تا معرفی سیستم طبقه‌بند انتخاب دستورالعمل براساس ورودی، همه و همه بیانگر این است که زاویه‌دید‌های مختلف نسبت به این مسئله باعث آشکار شدن ابعاد کشف‌نشده از مدل‌های زبانی بزرگ و رسیدن به دقت‌های بالاتر در درستی تشخیص خبر مهم از غیرمهم می‌شود.

همچنین این نکته را باید مدنظر گرفت که وظیفه تشخیص اهمیت یک خبر، بسیار کار پیچیده و مبهمی است زیرا که برای فرهنگ‌های مختلف، سلایق مختلف، شخصیت‌های مختلف و حتی دسته‌بندی‌های مختلف اهمیت یک خبر دچار دگرگونی است. در این مسئله حتی بحث زمان هم مطرح است یعنی یک خبر در یک حوزه زمانی می‌تواند مهم باشد در صورتی که برحه دیگه اهمیت خود را از دست بدهد.

در طول این پژوهش سعی شد با استفاده از دادگان برچسب‌گذاری جمع‌آوری‌شده از حدود ۱۱ هزار خبر تمامی نتایج گرفته‌شده و بررسی شود و مشخصا در یک زاویه دید دیگر و در دادگان دیگر ممکن است تعریف اهمیت خبر متفاوت باشد.

مسیری آینده این پژوهش می‌تواند به جست‌وجوی عمیق‌تر در خصوص سیستم‌های انتخاب دستعورالعمل‌ وابسته به ورودی کاربر بیانجامد خصوصا در حالت چندنمونه‌ای، به طوری که بالاترین دقت در این پژوهش از طریق این سیستم‌ها در حالت صفر‌نمونه گرفته شد و به نظر می‌رسد که این نوع سیستم‌ها می‌تواند در حالت‌های دیگر و به خصوص در حالت یادگیری چندنمونه‌ای به دقت‌های بالاتر دست یابد.

در نهایت این پروژه سعی کرده بسیاری از ابعاد استفاده از یک مدل زبانی بزرگ به عنوان یک تشخیص دهنده را بررسی کند و نتایج آن را اعلام کند به طوری که مسیر را برای پژوهش‌های آینده در حوزه استفاده از مدل‌های زبانی بزرگ در تشخیص «مهم» یا «غیرمهم» بودن یک خبر را هموارتر سازد.