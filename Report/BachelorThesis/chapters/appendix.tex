\فصل{مطالب تکمیلی}

در اینجا تمام محتوای تکمیلی پژوهش از جمله دستورالعمل‌های استفاده شده در مراحل مخلتف برای مدل‌های زبانی بزرگ و نتایج دسته‌بندی شده قرار گرفته شده است.

\قسمت{دستورالعمل‌های به کار گرفته شده}
در این بخش انواع دستورهای نوشته شده در این پژوهش آمده و شرح داده می‌شود.

\زیرقسمت{دستورهای وانیلا یا خام}
ابتدایی‌ترین دستورهای نوشته شده برای مدل‌های زبانی بزرگ به طوری که شامل تعریف اخبار مهم و شرح وظیفه است.

\vspace{5pt}
\begin{scriptsize}
\begin{itshape}
    هدف، داشتن یک دسته‌بند دودویی است که با گرفتن هر متن ورودی، کلاس آن را در خروجی مشخص می‌کند. کلاس‌ها شامل دو دسته‌ی غیرمهم و مهم هستند. یعنی خبر نوع غیرمهم و خبر نوع مهم.

    شرح تسک:
    
    متن یا خبری را مهم می‌گوییم اگر که برای بیش‌تر کاربران فارسی‌زبان اهمیت بالایی داشته باشد. یا به عبارت دیگر، جمعیت زیاد و بزرگی از ایرانیان مایل باشند که آن متن یا خبر را بخوانند و یا برای یکدیگر بفرستند.
    
     اگر خبری مربوط به یک قشر کوچک یا جامعه‌ی خاصی از کاربران باشد و یا ارزش خواندن کمی داشته باشد و یا خاص نباشد، آن خبر از نوع غیرمهم است.
    
    در صورتی که متن ورودی از نوع مهم باشد، کلاس مهم خواهد بود و در صورتی که غیرمهم باشد، کلاس غیرمهم خواهد بود.
    
    برخی از مفاهیم از نوع مهم عبارت‌اند از:
    یارانه و سهام و مواردی که قرار است پول به مردم برسد مهم هستند
    ثبت نام مسکن و خانه و اخبار مربوط به وام‌ها و...
    ثبت نام خودرو
    افزایش و کاهش های شدید و زیاد قیمت ارز یا طلا و سکه و یا تورم
    
    سیاسی:
    اخبار جنگ، برجام، توافق های ایران،
    تحریم های ایران،
    خبرهای جنگ‌های بزرگ منطقه‌ای،
    عزل و نصب مقامات بلندپایه ایرانی،
    این‌ها همگی مهم هستند
    
    ورزشی:
    اخبار مربوط به تیم‌های معروف و پرطرفدار ایرانی و همین‌طور اروپایی مهم است
    
    تمام اخبار بالا از نوع مهم بوده و اخبار دسته‌های دیگر که کمتر خواننده دارند را از نوع غیرمهم در نظر می‌گیریم.
    
    با توجه به متن زیر تنها در یک عدد پاسخ بده که باتوجه به مفاهیمی که در بالا مطرح شد و قدرت استنتاجی که خودت داری، آیا متن مهم حساب می‌شود یا غیرمهم. (مهم یا غیرمهم):
    
    در خروجی فقط مجاز هستی مهم یا عدد غیرمهم بنویسی. بدون هیچ توضیح اضافه‌ای.
\end{itshape}
\end{scriptsize}
\vspace{5pt}

و نسخه انگلیسی آن به این صورت نوشته و مورد بررسی قرار گرفته است:

\vspace{5pt}
\begin{scriptsize}
\begin{itshape}
\begin{latin}
\LTR
The goal is to have a binary classifier that, by receiving any input text, determines its class in the output. The classes include two categories: 'not important' and 'important', meaning news type 'not important' and news type 'important'.

Task description:

We label a text or news as 'important' if it is of high importance to most Persian-speaking users. In other words, if a large population of Iranians are likely to read, share, or be interested in it, it is classified as 'important'.

If the news pertains to a small group or a specific community of users, has little reading value, or is not significant, it is classified as 'not important'.

If the input text is of type 'important', the output class will be 'important'; if it is 'not important', the output class will be 'not important'.

Some concepts that fall under type 'important' are:
Subsidies, stocks, and matters that involve receiving money are important.
Housing and home registrations, news related to loans, etc.
Car registrations
Significant fluctuations in currency, gold, coins, or inflation rates

Politics:
News about war, the JCPOA, Iran’s agreements,
Sanctions on Iran,
News of major regional wars,
Dismissal and appointment of high-ranking Iranian officials,
These are all important.

Sports:
News about famous and popular Iranian teams as well as European teams is important.

All the above news are classified as type 'important', and other news categories that have fewer readers are considered as type 'not important'.

A text or news is classified as 'not important' if it pertains to a specific small section of the society. News that does not engage a broad spectrum of the community is type 'not important'. For example:
Sports: News about non-famous clubs and small events are of type 'not important'.
Politics: News about non-prominent figures that do not affect the Iranian society is of type 'not important'.
Social: News that does not engage a large section of society is type 'not important'.

Based on the following text, respond with only a single label that, considering the concepts discussed above and your own inferential ability, indicates whether the text should be classified as 'important' or 'not important'. ('important' or 'not important'):

You are only allowed to write the label 'important' or 'not important' in the output, without any additional explanation.
\RTL
\end{latin}
\end{itshape}
\end{scriptsize}
\vspace{5pt}

که شامل تعاریف اخبار مهم برای دسته‌های مختلف خبری به صورت اضافه‌تر است.

\زیرقسمت{دستورهای نمادین}
در اینجا در دستور نوشته شده، هیچ اسمی از «مهم» و «غیرمهم» بودن و تعاریف آنها برده نشده و صرفا از برچسب‌های نمادین «۵۸» و «۴۷» استفاده شده است.

\vspace{5pt}
\begin{scriptsize}
\begin{itshape}
    هدف، داشتن یک دسته‌بند دودویی است که با گرفتن هر متن ورودی، کلاس آن را در خروجی مشخص می‌کند. کلاس‌ها شامل دو دسته‌ی 47 و 58 هستند. یعنی خبر نوع 47 و خبر نوع 58.

    شرح تسک:
    
    متن یا خبری را 58 می‌گوییم اگر که برای بیش‌تر کاربران فارسی‌زبان اهمیت بالایی داشته باشد. یا به عبارت دیگر، جمعیت زیاد و بزرگی از ایرانیان مایل باشند که آن متن یا خبر را بخوانند و یا برای یکدیگر بفرستند.
    
     اگر خبری مربوط به یک قشر کوچک یا جامعه‌ی خاصی از کاربران باشد و یا ارزش خواندن کمی داشته باشد و یا خاص نباشد، آن خبر از نوع 47 است.
    
    در صورتی که متن ورودی از نوع 58 باشد، کلاس 58 خواهد بود و در صورتی که 47 باشد، کلاس 47 خواهد بود.
    
    برخی از مفاهیم از نوع 58 عبارت‌اند از:
    یارانه و سهام و مواردی که قرار است پول به مردم برسد مهم هستند
    ثبت نام مسکن و خانه و اخبار مربوط به وام‌ها و...
    ثبت نام خودرو
    افزایش و کاهش های شدید و زیاد قیمت ارز یا طلا و سکه و یا تورم
    
    سیاسی:
    اخبار جنگ، برجام، توافق های ایران،
    تحریم های ایران،
    خبرهای جنگ‌های بزرگ منطقه‌ای،
    عزل و نصب مقامات بلندپایه ایرانی،
    این‌ها همگی مهم هستند
    
    ورزشی:
    اخبار مربوط به تیم‌های معروف و پرطرفدار ایرانی و همین‌طور اروپایی مهم است
    
    تمام اخبار بالا از نوع 58 بوده و اخبار دسته‌های دیگر که کمتر خواننده دارند را از نوع 47 در نظر می‌گیریم.
    
    با توجه به متن زیر تنها در یک عدد پاسخ بده که باتوجه به مفاهیمی که در بالا مطرح شد و قدرت استنتاجی که خودت داری، آیا متن 58 حساب می‌شود یا 47. (58 یا 47):
    
    در خروجی فقط مجاز هستی عدد 58 یا عدد 47 بنویسی. بدون هیچ توضیح اضافه‌ای.
\end{itshape}
\end{scriptsize}
\vspace{5pt}

در نتایج به دست آمده، متوجه شدیم که در این دستور صرفا تعریف اخبار «مهم» یا همان «۵۸» آمده است و برای همین مدل در رویکر بدون نمونه صرفا تمامی اخبار را مهم پیش‌بینی می‌کند. برای حل این مشکل دستور زیر با اضافه شدن تعاریف اخبار غیرمهم یا همان «۴۷» نوشته و به کار گرفته شد.

\vspace{5pt}
\begin{scriptsize}
\begin{itshape}
    هدف، داشتن یک دسته‌بند دودویی است که با گرفتن هر متن ورودی، کلاس آن را در خروجی مشخص می‌کند. کلاس‌ها شامل دو دسته‌ی 47 و 58 هستند. یعنی خبر نوع 47 و خبر نوع 58.

    شرح تسک:
    
    متن یا خبری را 58 می‌گوییم اگر که برای بیش‌تر کاربران فارسی‌زبان اهمیت بالایی داشته باشد. یا به عبارت دیگر، جمعیت زیاد و بزرگی از ایرانیان مایل باشند که آن متن یا خبر را بخوانند و یا برای یکدیگر بفرستند.
    
     اگر خبری مربوط به یک قشر کوچک یا جامعه‌ی خاصی از کاربران باشد و یا ارزش خواندن کمی داشته باشد و یا خاص نباشد، آن خبر از نوع 47 است.
    
    در صورتی که متن ورودی از نوع 58 باشد، کلاس 58 خواهد بود و در صورتی که 47 باشد، کلاس 47 خواهد بود.
    
    برخی از مفاهیم از نوع 58 عبارت‌اند از:
    یارانه و سهام و مواردی که قرار است پول به مردم برسد مهم هستند
    ثبت نام مسکن و خانه و اخبار مربوط به وام‌ها و...
    ثبت نام خودرو
    افزایش و کاهش های شدید و زیاد قیمت ارز یا طلا و سکه و یا تورم
    
    سیاسی:
    اخبار جنگ، برجام، توافق های ایران،
    تحریم های ایران،
    خبرهای جنگ‌های بزرگ منطقه‌ای،
    عزل و نصب مقامات بلندپایه ایرانی،
    این‌ها همگی مهم هستند
    
    ورزشی:
    اخبار مربوط به تیم‌های معروف و پرطرفدار ایرانی و همین‌طور اروپایی مهم است
    
    تمام اخبار بالا از نوع 58 بوده و اخبار دسته‌های دیگر که کمتر خواننده دارند را از نوع 47 در نظر می‌گیریم.
    
    متن یا خبری را 47 می‌گویند که مربوط به بخش خاص و کوچکی از جامعه باشد. اخباری که گستره‌ی وسیعی از جامعه را درگیر نکند، اخبار از نوع 47 هستند. برای نمونه:
    ورزشی: اخبار مربوط به باشگاه‌های غیر معروف و رخدادهای کوچک از نوع 47 هستند.
    سیاسی: اخبار مربوط به شخصیت‌های غیرمشهور که تاثیری روی جامعه‌ی ایران ندارد از نوع 47 هستند.
    اجتماعی: اخباری که گستره‌ی وسیعی از جامعه را درگیر نمی‌کند از نوع 47هستند.
    
    با توجه به متن زیر تنها در یک عدد پاسخ بده که باتوجه به مفاهیمی که در بالا مطرح شد و قدرت استنتاجی که خودت داری، آیا متن 58 حساب می‌شود یا 47. (58 یا 47):
    
    در خروجی فقط مجاز هستی عدد 58 یا عدد 47 بنویسی. بدون هیچ توضیح اضافه‌ای.
\end{itshape}
\end{scriptsize}
\vspace{5pt}

\زیرقسمت{دستورها با رویکرد یادگیری چند نمونه‌ای}
در این نوع پرامپت‌ها، چندین نمونه و مثال برای یادگیری و شباهت‌سنجی در اختیار مدل زبانی بزرگ قرار می‌گیرد که یک نمونه از این نوع دستورها در اینجا قرار داده شده است.

\vspace{5pt}
\begin{scriptsize}
\begin{itshape}
    هدف، داشتن یک دسته‌بند دودویی است که با گرفتن هر متن ورودی، کلاس آن را در خروجی مشخص می‌کند. کلاس‌ها شامل دو دسته‌ی 1 یا 0 هستند. 1 یعنی خبر مهم است و 0 یعنی خبر مهم نیست.

شرح تسک:

متن یا خبری را مهم یا تاثیرگذار می‌گوییم اگر که برای بیش‌تر کاربران فارسی‌زبان اهمیت بالایی داشته باشد. یا به عبارت دیگر، جمعیت زیاد و بزرگی از ایرانیان مایل باشند که آن متن یا خبر را بخوانند و یا برای یکدیگر بفرستند. اگر خبری مربوط به یک قشر کوچک یا جامعه‌ی خاصی از کاربران باشد، آن خبر مهم نیست.
در صورتی که متن ورودی مهم باشد، کلاس 1 خواهد بود و در صورتی که مهم نباشد، کلاس 0 خواهد بود

برخی از مفاهیم مهم و از کلاس 1 عبارت‌اند از:
یارانه و سهام و مواردی که قرار است پول به مردم برسد مهم هستند
ثبت نام مسکن و خانه و اخبار مربوط به وام‌ها و... 
ثبت نام خودرو
افزایش و کاهش های شدید و زیاد قیمت ارز یا طلا و سکه و یا تورم 

سیاسی:
اخبار جنگ، برجام، توافق های ایران، 
تحریم های ایران، 
خبرهای جنگ‌های بزرگ منطقه‌ای،
عزل و نصب مقامات بلندپایه ایرانی،
این‌ها همگی مهم هستند

ورزشی:
اخبار مربوط به تیم‌های معروف و پرطرفدار ایرانی و همین‌طور اروپایی مهم است


نمونه‌ها: چند نمونه پایین را ببین و باتوجه به آن‌ها به سوال پایین پاسخ بده

SAMPLES

از روی نمونه‌های بالایی یاد بگیر و خروجی را مشخص کن (فقط 0 یا 1).
حال  با توجه به «نمونه‌های بالا»، برای متن زیر تنها در یک واژه پاسخ بده که باتوجه به مفاهیمی که در بالا مطرح شد و قدرت استنتاجی که خودت داری، آیا متن 
مهم (تاثیرگذاری) حساب می‌شود یا خیر. (1 یا 0):

در خروجی فقط مجاز هستی عدد 1 یا عدد 0 بنویسی. بدون هیچ توضیح اضافه‌ای.
\end{itshape}
\end{scriptsize}
\vspace{5pt}





\قسمت{نتایج اضافی‌تر}

در این بخش به نتایج اضافه‌تر در دسته‌های گوناگون خبری از جمله سیاسی، اقتصادی، وزشی و فرهنگی اشاره شده است.