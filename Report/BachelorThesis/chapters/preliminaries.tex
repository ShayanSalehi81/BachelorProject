\فصل{مفاهیم اولیه}
در اینجا به مفاهیم اصلی به کار برده شده در این پروژه، و بررسی کاربرد و پیشینه آن می‌پردازیم.

\قسمت{مدل‌های زبانی بزرگ}
مدل‌های زبانی بزرگ\پانویس{Large Language Models} به سامانه‌های هوش مصنوعی گفته می‌شوند که بر اساس پردازش زبان طبیعی طراحی شده‌اند و قادر به تولید و درک متن‌های انسانی در مقیاس وسیع هستند. این مدل‌ها با استفاده از حجم بسیار زیادی از داده‌های متنی آموزش می‌بینند و می‌توانند وظایف متنوع زبانی، از جمله ترجمه، خلاصه‌سازی و پاسخ به پرسش‌ها را انجام دهند. 

مفهوم مدل‌های زبانی از دهه ۱۹۸۰ با ظهور الگوریتم‌های احتمالاتی ساده آغاز شد. با معرفی شبکه‌های عصبی در دهه ۱۹۹۰ و توسعه یادگیری عمیق در دهه ۲۰۱۰، مدل‌هایی مانند ترانسفورمرها و سیستم‌هایی نظیر جی‌پی‌تی (نسل اول تا سوم) و مدل‌های مشابه توانستند به کارایی فوق‌العاده‌ای دست یابند.\مرجع{devlin2019bertpretrainingdeepbidirectional}
افزایش قدرت محاسباتی و دسترسی به داده‌های بیشتر، این پیشرفت‌ها را تسهیل کرد.

\قسمت{یادگیری درونی}
یادگیری درون‌متنی\پانویس{In-Context Learning} به توانایی یک مدل زبانی اشاره دارد که بتواند بر اساس نمونه‌هایی که در همان متن ورودی ارائه می‌شود، وظایف جدیدی را یاد بگیرد. در این روش، نیاز به آموزش دوباره مدل وجود ندارد، بلکه مدل از اطلاعات داده شده در همان لحظه استفاده می‌کند.\مرجع{brown2020languagemodelsfewshotlearners}

این مفهوم در اوایل دهه ۲۰۲۰ با توسعه مدل‌هایی مانند جی‌پی‌تی ۳ به وضوح مطرح شد. این مدل‌ها نشان دادند که بدون نیاز به آموزش دوباره، می‌توانند تنها با ارائه نمونه‌هایی در ورودی، وظایف مختلفی را انجام دهند. این پیشرفت‌ها نقطه عطفی در ساده‌سازی استفاده از مدل‌های زبانی محسوب می‌شوند.

\قسمت{تنظیم بر اساس دستورالعمل}
تنظیم بر اساس دستورالعمل\پانویس{Insturction Tuning} فرآیندی است که در آن یک مدل هوش مصنوعی با استفاده از داده‌هایی آموزش می‌بیند که حاوی دستورالعمل‌های خاصی برای انجام وظایف مختلف هستند.\مرجع{wei2022finetunedlanguagemodelszeroshot}
هدف این روش بهبود عملکرد مدل در درک و اجرای دستورالعمل‌هاست.

ایده این روش از مفاهیم یادگیری انتقالی نشأت گرفته است. در سال‌های اخیر، با توجه به توانایی مدل‌های بزرگ زبانی در تعمیم وظایف، محققان تلاش کردند تا این مدل‌ها را با داده‌های حاوی دستورالعمل بهبود دهند. پروژه‌هایی مانند اجرای دستور عمل در مدل‌های جی‌پی‌تی\مرجع{ouyang2022traininglanguagemodelsfollow} نشان‌دهنده موفقیت این رویکرد هستند.

\قسمت{درخواست‌های سامانه و کاربر}
درخواست‌های سامانه\پانویس{System Prompt} و کاربر به متونی اطلاق می‌شود که برای هدایت مدل زبانی به سمت تولید پاسخ مناسب استفاده می‌شوند. درخواست سامانه معمولاً وظیفه مشخص‌کردن قواعد کلی را دارد، در حالی که درخواست کاربر هدف یا سؤال خاصی را بیان می‌کند.\مرجع{gao2019dialogstatetrackingneural}

این مفهوم با گسترش استفاده از مدل‌های زبانی در تعاملات انسانی به وجود آمد. اولین تلاش‌ها برای تعریف و تمایز این دو نوع درخواست در توسعه رابط‌های کاربری تعاملی و چت‌بات‌ها مشاهده شد. این ایده در مدل‌های زبانی بزرگ تکامل یافت.

\قسمت{مهندسی درخواست}
مهندسی درخواست\پانویس{Prompt Engineering} به هنر و دانش طراحی درخواست‌ها برای هدایت مدل‌های زبانی جهت تولید پاسخ‌های دقیق و مفید اشاره دارد. این فرآیند شامل ایجاد ورودی‌هایی است که بتوانند بهترین نتیجه ممکن را از مدل دریافت کنند.

این مفهوم با ظهور مدل‌های زبانی پیچیده و نیاز به بهره‌برداری بهتر از توانایی‌های آن‌ها مطرح شد. در سال‌های اخیر، مقالات و ابزارهای بسیاری برای استانداردسازی و بهبود این فرآیند ارائه شده است.\مرجع{reynolds2021promptprogramminglargelanguage} مهندسی درخواست در زمینه‌های مختلف، از پژوهش گرفته تا صنعت، نقش کلیدی ایفا می‌کند.


























































% دومین فصل پایان‌نامه به طور معمول به معرفی مفاهیمی می‌پردازد که در پایان‌نامه مورد استفاده قرار می‌گیرند.
% در این فصل به عنوان یک نمونه، نکات کلی در خصوص نحوه‌ی نگارش پایان‌نامه
% و نیز برخی نکات نگارشی به اختصار توضیح داده می‌شوند.

% \قسمت{نحوه‌ی نگارش}

% \زیرقسمت{پرونده‌ها}

% پرونده‌ی اصلی پایان‌نامه در قالب استاندارد\زیرنویس{
% قالب استاندارد پایان‌نامه از نشانی
% \href{https://github.com/zarrabi/thesis-template}
% {github.com/zarrabi/thesis-template}
% قابل دریافت است.}
% \کد{thesis.tex}  نام دارد.
% به ازای هر فصل از پایان‌نامه، یک پرونده در شاخه‌ی \کد{chapters} ایجاد نموده
% و نام آن را در  \کد{thesis.tex} (در قسمت فصل‌ها) درج نمایید.
% برای مشاهده‌ی خروجی، پرونده‌ی \کد{thesis.tex} را با زی‌لاتک کامپایل کنید.
% مشخصات اصلی پایان‌نامه را می‌توانید در پرونده‌ی \کد{front/info.tex} ویرایش کنید.

% \زیرقسمت{عبارات ریاضی}

% برای درج عبارات ریاضی در داخل متن از \$...\$ و 
% برای درج عبارات ریاضی در یک خط مجزا از \$\$...\$\$ یا محیط \لر{equation} 
% استفاده کنید. برای مثال عبارت 
% $2x + 3y$
% در داخل متن و عبارت زیر
% \begin{equation}
% \sum_{k=0}^{n} \binom{n}{k} = 2^n
% \end{equation}
% در یک خط مجزا درج شده است. 
% دقت کنید که تمامی عبارات ریاضی، از جمله متغیرهای تک‌حرفی مانند $x$ و $y$ باید در محیط ریاضی 
% یعنی محصور بین دو علامت \$ باشند. 


% \زیرقسمت{علائم ریاضی پرکاربرد}

% برخی علائم ریاضی پرکاربرد در زیر فهرست شده‌اند. 
% برای مشاهده‌ی دستور  معادل پرونده‌ی منبع را ببینید.


% \شروع{فقرات}
% \فقره مجموعه‌‌های اعداد: 
% $\IN, \IZ, \IZ^+, \IQ, \IR, \IC$
% \فقره مجموعه:
% $\set{1, 2, 3}$
% \فقره دنباله‌:
% $\seq{1, 2, 3}$
% \فقره سقف و کف:
% $\ceil{x}, \floor{x}$
% \فقره اندازه و متمم:
% $\card{A}, \setcomp{A}$
% \فقره همنهشتی:
% $a \iequiv{n} 1$
% یا
% $a \equiv 1 \imod{n}$ 
% %\فقره شمردن (عاد کردن):
% %$3 \divs n, 2 \ndivs n$
% \فقره ضرب و تقسیم:
% $\times, \cdot, \div$
% \فقره سه‌نقطه‌:
% $1, 2, \dots, n$
% \فقره کسر و ترکیب:
% $\frac{n}{k}, \binom{n}{k}$
% \فقره اجتماع و اشتراک:
% $A \cup (B \cap C)$
% \فقره عملگرهای منطقی:
% $\neg p \vee (q \wedge r)$

% \فقره پیکان‌ها:
% $\rightarrow, \Rightarrow, \leftarrow, \Leftarrow, \leftrightarrow, \Leftrightarrow$
% \فقره عملگرهای مقایسه‌ای:
% $\not=, \le, \not\le, \ge, \not\ge$
% \فقره عملگرهای مجموعه‌ای:
% $\in, \not\in, \setminus, \subset, \subseteq, \subsetneq, \supset, \supseteq, \supsetneq$

% \فقره جمع و ضرب چندتایی:
% $\sum_{i=1}^{n} a_i, \prod_{i=1}^{n} a_i$
% \فقره اجتماع و اشتراک چندتایی:
% $\bigcup_{i=1}^{n} A_i, \bigcap_{i=1}^{n} A_i$
% \فقره برخی نمادها:
% $\infty, \emptyset, \forall, \exists, \triangle, \angle, \ell, \equiv, \therefore$
% \پایان{فقرات}


% \زیرقسمت{لیست‌ها}

% برای ایجاد یک لیست‌ می‌توانید از محیط‌های «فقرات» و «شمارش» همانند زیر استفاده کنید.

% \begin{multicols}{2}
% \شروع{فقرات}
% \فقره مورد اول
% \فقره مورد دوم
% \فقره مورد سوم
% \پایان{فقرات}

% \شروع{شمارش}
% \فقره مورد اول
% \فقره مورد دوم
% \فقره مورد سوم
% \پایان{شمارش}

% \end{multicols}


% \زیرقسمت{درج شکل}

% یکی از روش‌های مناسب برای ایجاد شکل استفاده از نرم‌افزار \لر{LaTeX Draw} و سپس
% درج خروجی آن به صورت یک فایل \کد{tex} درون متن 
% با استفاده از دستور  \کد{fig} یا \کد{centerfig} است.
% شکل~\رجوع{شکل:پوشش رأسی} نمونه‌ای از اشکال ایجادشده با این ابزار را نشان می‌دهد.


% \شروع{شکل}[ht]
% \centerfig{cover.tex}{.9}
% \شرح{یک گراف و پوشش رأسی آن}
% \برچسب{شکل:پوشش رأسی}
% \پایان{شکل}

% \bigskip
% همچنین می‌توانید با استفاده از نرم‌افزار \lr{Ipe} شکل‌های خود را مستقیما
% به صورت \لر{pdf} ایجاد نموده و آن‌ها را با دستورات \کد{img} یا  \کد{centerimg} 
% درون متن درج کنید. برای نمونه، شکل~\رجوع{شکل:گراف جهت‌دار} را ببینید.


% \شروع{شکل}[ht]
% \centerimg{strip}{6.5cm}
% \شرح{نمونه شکل ایجادشده توسط نرم‌افزار \lr{Ipe}}
% \برچسب{شکل:گراف جهت‌دار}
% \پایان{شکل}


% \زیرقسمت{درج جدول}

% برای درج جدول می‌توانید با استفاده از دستور  «جدول»
% جدول را ایجاد کرده و سپس با دستور  «لوح»  آن را درون متن درج کنید.
% برای نمونه جدول~\رجوع{جدول:عملگرهای مقایسه‌ای} را ببینید.

% \vspace{1.5em}

% \شروع{لوح}[ht]
% \تنظیم‌ازوسط
% \شرح{عملگرهای مقایسه‌ای}

% \شروع{جدول}{|c|c|}
% \خط‌پر 
% \سیاه عملگر & \سیاه عنوان \\ 
% \خط‌پر \خط‌پر 
% \کد{<} & کوچک‌تر \\ 
% \کد{>} & بزرگ‌تر \\
% \کد{==} &  مساوی \\ 
% \کد{<>} & نامساوی \\ 
% \خط‌پر
% \پایان{جدول}

% \برچسب{جدول:عملگرهای مقایسه‌ای}
% \پایان{لوح}



% \زیرقسمت{درج الگوریتم}

% برای درج الگوریتم می‌توانید از محیط «الگوریتم» استفاده کنید.
% یک نمونه در الگوریتم~\رجوع{الگوریتم: پوشش رأسی حریصانه} آمده است.

% \شروع{الگوریتم}{پوشش رأسی حریصانه}
% \ورودی گراف $G=(V, E)$
% \خروجی یک پوشش رأسی از $G$

% \دستور قرار بده $C = \emptyset$  % \توضیحات{مقداردهی اولیه}
% \تاوقتی{$E$ تهی نیست}
% %\اگر{$|E| > 0$}
% %	\دستور{یک کاری انجام بده}
% %\پایان‌اگر
% \دستور یال دل‌‌خواه $uv \in E$ را انتخاب کن
% \دستور رأس‌های $u$ و $v$ را به $C$ اضافه کن
% \دستور تمام یال‌های واقع بر $u$ یا $v$ را از $E$ حذف کن
% \پایان‌تاوقتی
% \دستور $C$ را برگردان
% \پایان{الگوریتم}


% \زیرقسمت{محیط‌های ویژه}

% برای درج مثال‌ها، قضیه‌ها، لم‌ها و نتیجه‌ها به ترتیب از محیط‌های
% «مثال»، «قضیه»، «لم» و «نتیجه» استفاده کنید.
% برای درج اثبات قضیه‌ها و لم‌ها  از محیط «اثبات» استفاده کنید.

% تعریف‌های داخل متن را با استفاده از دستور «مهم» به صورت \مهم{تیره‌} نشان دهید.
% تعریف‌های پایه‌ای‌تر را درون محیط «تعریف» قرار دهید.

% \شروع{تعریف}[اصل لانه‌کبوتری]
% اگر $n+1$ کبوتر یا بیش‌تر درون  $n$ لانه قرار گیرند، آن‌گاه لانه‌ای 
% وجود دارد که شامل حداقل دو کبوتر است.
% \پایان{تعریف}




% \قسمت{برخی نکات نگارشی}

% این فصل حاوی برخی نکات ابتدایی ولی بسیار مهم در نگارش متون فارسی است. 
% نکات گردآوری‌شده در این فصل به‌ هیچ‌ وجه کامل نیست، 
% ولی دربردارنده‌ی حداقل مواردی است که رعایت آن‌ها در نگارش پایان‌نامه ضروری به نظر می‌رسد.

% \زیرقسمت{فاصله‌گذاری}

% \شروع{شمارش}

% \فقره 
% علائم سجاوندی مانند نقطه، ویرگول، دونقطه، نقطه‌ویرگول، علامت سؤال و علامت تعجب % (. ، : ؛ ؟ !) 
% بدون فاصله از کلمه‌ی پیشین خود نوشته می‌شوند، ولی بعد از آن‌ها باید یک فاصله‌ قرار گیرد. مانند: من، تو، او.
% \فقره 
% علامت‌های پرانتز، آکولاد، کروشه، نقل قول و نظایر آن‌ها بدون فاصله با عبارات داخل خود نوشته می‌شوند، ولی با عبارات اطراف خود یک فاصله دارند. مانند: (این عبارت) یا \{آن عبارت\}.
% \فقره 
% دو کلمه‌ی متوالی در یک جمله همواره با یک فاصله از هم جدا می‌شوند، ولی اجزای یک کلمه‌ی مرکب باید با نیم‌فاصله\زیرنویس{«نیم‌فاصله» فاصله‌‌ای مجازی است که در عین جدا کردن اجزای یک کلمه‌ی مرکب از یک‌دیگر، آن‌ها را نزدیک به هم نگه می‌دارد. معمولاً برای تولید این نوع فاصله در صفحه‌کلید‌های استاندارد از ترکیب Shift+Space استفاده می‌شود.}‌‌
%  از هم جدا شوند. مانند: کتاب درس، محبت‌آمیز، دوبخشی.
%  \فقره 
%  اجزای فعل‌های مرکب با فاصله از یک‌دیگر نوشته می‌شوند، مانند: تحریر کردن، به سر آمدن.
% \پایان{شمارش}