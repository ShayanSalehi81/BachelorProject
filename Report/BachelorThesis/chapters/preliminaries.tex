\فصل{مفاهیم اولیه}
در اینجا به مفاهیم اصلی به کار برده شده در این پروژه، و بررسی کاربرد و پیشینه آن می‌پردازیم.

\قسمت{مدل‌های زبانی بزرگ}
مدل‌های زبانی بزرگ\پانویس{Large Language Models} به سامانه‌های هوش مصنوعی گفته می‌شوند که بر اساس پردازش زبان طبیعی طراحی شده‌اند و قادر به تولید و درک متن‌های انسانی در مقیاس وسیع هستند. این مدل‌ها با استفاده از حجم بسیار زیادی از داده‌های متنی آموزش می‌بینند و می‌توانند وظایف متنوع زبانی، از جمله ترجمه، خلاصه‌سازی و پاسخ به پرسش‌ها را انجام دهند. 

مفهوم مدل‌های زبانی از دهه ۱۹۸۰ با ظهور الگوریتم‌های احتمالاتی ساده آغاز شد. با معرفی شبکه‌های عصبی در دهه ۱۹۹۰ و توسعه یادگیری عمیق در دهه ۲۰۱۰، مدل‌هایی مانند ترانسفورمرها و سیستم‌هایی نظیر جی‌پی‌تی (نسل اول تا سوم) و مدل‌های مشابه توانستند به کارایی فوق‌العاده‌ای دست یابند.\مرجع{devlin2019bertpretrainingdeepbidirectional}
افزایش قدرت محاسباتی و دسترسی به داده‌های بیشتر، این پیشرفت‌ها را تسهیل کرد.

\قسمت{یادگیری درونی}
یادگیری درون‌متنی\پانویس{In-Context Learning} به توانایی یک مدل زبانی اشاره دارد که بتواند بر اساس نمونه‌هایی که در همان متن ورودی ارائه می‌شود، وظایف جدیدی را یاد بگیرد. در این روش، نیاز به آموزش دوباره مدل وجود ندارد، بلکه مدل از اطلاعات داده شده در همان لحظه استفاده می‌کند.\مرجع{brown2020languagemodelsfewshotlearners}

این مفهوم در اوایل دهه ۲۰۲۰ با توسعه مدل‌هایی مانند جی‌پی‌تی ۳ به وضوح مطرح شد. این مدل‌ها نشان دادند که بدون نیاز به آموزش دوباره، می‌توانند تنها با ارائه نمونه‌هایی در ورودی، وظایف مختلفی را انجام دهند. این پیشرفت‌ها نقطه عطفی در ساده‌سازی استفاده از مدل‌های زبانی محسوب می‌شوند.

\قسمت{تنظیم بر اساس دستورالعمل}
تنظیم بر اساس دستورالعمل\پانویس{Insturction Tuning} فرآیندی است که در آن یک مدل هوش مصنوعی با استفاده از داده‌هایی آموزش می‌بیند که حاوی دستورالعمل‌های خاصی برای انجام وظایف مختلف هستند.\مرجع{wei2022finetunedlanguagemodelszeroshot}
هدف این روش بهبود عملکرد مدل در درک و اجرای دستورالعمل‌هاست.

ایده این روش از مفاهیم یادگیری انتقالی نشأت گرفته است. در سال‌های اخیر، با توجه به توانایی مدل‌های بزرگ زبانی در تعمیم وظایف، محققان تلاش کردند تا این مدل‌ها را با داده‌های حاوی دستورالعمل بهبود دهند. پروژه‌هایی مانند اجرای دستور عمل در مدل‌های جی‌پی‌تی\مرجع{ouyang2022traininglanguagemodelsfollow} نشان‌دهنده موفقیت این رویکرد هستند.

\قسمت{درخواست‌های سامانه و کاربر}
درخواست‌های سامانه\پانویس{System Prompt} و کاربر به متونی اطلاق می‌شود که برای هدایت مدل زبانی به سمت تولید پاسخ مناسب استفاده می‌شوند. درخواست سامانه معمولاً وظیفه مشخص‌کردن قواعد کلی را دارد، در حالی که درخواست کاربر هدف یا سؤال خاصی را بیان می‌کند.\مرجع{gao2019dialogstatetrackingneural}

این مفهوم با گسترش استفاده از مدل‌های زبانی در تعاملات انسانی به وجود آمد. اولین تلاش‌ها برای تعریف و تمایز این دو نوع درخواست در توسعه رابط‌های کاربری تعاملی و چت‌بات‌ها مشاهده شد. این ایده در مدل‌های زبانی بزرگ تکامل یافت.

\قسمت{مهندسی درخواست}
مهندسی درخواست\پانویس{Prompt Engineering} به هنر و دانش طراحی درخواست‌ها برای هدایت مدل‌های زبانی جهت تولید پاسخ‌های دقیق و مفید اشاره دارد. این فرآیند شامل ایجاد ورودی‌هایی است که بتوانند بهترین نتیجه ممکن را از مدل دریافت کنند.

این مفهوم با ظهور مدل‌های زبانی پیچیده و نیاز به بهره‌برداری بهتر از توانایی‌های آن‌ها مطرح شد. در سال‌های اخیر، مقالات و ابزارهای بسیاری برای استانداردسازی و بهبود این فرآیند ارائه شده است.\مرجع{reynolds2021promptprogramminglargelanguage} مهندسی درخواست در زمینه‌های مختلف، از پژوهش گرفته تا صنعت، نقش کلیدی ایفا می‌کند.